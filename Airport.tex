\documentclass[]{article}
\usepackage{lmodern}
\usepackage{amssymb,amsmath}
\usepackage{ifxetex,ifluatex}
\usepackage{fixltx2e} % provides \textsubscript
\ifnum 0\ifxetex 1\fi\ifluatex 1\fi=0 % if pdftex
  \usepackage[T1]{fontenc}
  \usepackage[utf8]{inputenc}
\else % if luatex or xelatex
  \ifxetex
    \usepackage{mathspec}
  \else
    \usepackage{fontspec}
  \fi
  \defaultfontfeatures{Ligatures=TeX,Scale=MatchLowercase}
\fi
% use upquote if available, for straight quotes in verbatim environments
\IfFileExists{upquote.sty}{\usepackage{upquote}}{}
% use microtype if available
\IfFileExists{microtype.sty}{%
\usepackage{microtype}
\UseMicrotypeSet[protrusion]{basicmath} % disable protrusion for tt fonts
}{}
\usepackage[margin=1in]{geometry}
\usepackage{hyperref}
\hypersetup{unicode=true,
            pdfborder={0 0 0},
            breaklinks=true}
\urlstyle{same}  % don't use monospace font for urls
\usepackage{graphicx,grffile}
\makeatletter
\def\maxwidth{\ifdim\Gin@nat@width>\linewidth\linewidth\else\Gin@nat@width\fi}
\def\maxheight{\ifdim\Gin@nat@height>\textheight\textheight\else\Gin@nat@height\fi}
\makeatother
% Scale images if necessary, so that they will not overflow the page
% margins by default, and it is still possible to overwrite the defaults
% using explicit options in \includegraphics[width, height, ...]{}
\setkeys{Gin}{width=\maxwidth,height=\maxheight,keepaspectratio}
\IfFileExists{parskip.sty}{%
\usepackage{parskip}
}{% else
\setlength{\parindent}{0pt}
\setlength{\parskip}{6pt plus 2pt minus 1pt}
}
\setlength{\emergencystretch}{3em}  % prevent overfull lines
\providecommand{\tightlist}{%
  \setlength{\itemsep}{0pt}\setlength{\parskip}{0pt}}
\setcounter{secnumdepth}{0}
% Redefines (sub)paragraphs to behave more like sections
\ifx\paragraph\undefined\else
\let\oldparagraph\paragraph
\renewcommand{\paragraph}[1]{\oldparagraph{#1}\mbox{}}
\fi
\ifx\subparagraph\undefined\else
\let\oldsubparagraph\subparagraph
\renewcommand{\subparagraph}[1]{\oldsubparagraph{#1}\mbox{}}
\fi

%%% Use protect on footnotes to avoid problems with footnotes in titles
\let\rmarkdownfootnote\footnote%
\def\footnote{\protect\rmarkdownfootnote}

%%% Change title format to be more compact
\usepackage{titling}

% Create subtitle command for use in maketitle
\providecommand{\subtitle}[1]{
  \posttitle{
    \begin{center}\large#1\end{center}
    }
}

\setlength{\droptitle}{-2em}

  \title{}
    \pretitle{\vspace{\droptitle}}
  \posttitle{}
    \author{}
    \preauthor{}\postauthor{}
    \date{}
    \predate{}\postdate{}
  

\begin{document}

\hypertarget{question-1}{%
\section{Question 1}\label{question-1}}

\hypertarget{isaac-hulsey-idh285}{%
\subsection{Isaac Hulsey idh285}\label{isaac-hulsey-idh285}}

\hypertarget{im-interested-in-whether-or-not-theres-an-airport-you-dont-want-your-flight-from-austin-to-come-from-in-regards-to-average-delay-time.-so-first-i-created-a-simple-scatterplot-between-airport-of-origin-and-delay-time-to-see-if-any-trends-existed.}{%
\subsubsection{I'm interested in whether or not there's an airport you
don't want your flight from Austin to come from in regards to average
delay time. So, first I created a simple scatterplot between airport of
origin and delay time to see if any trends
existed.}\label{im-interested-in-whether-or-not-theres-an-airport-you-dont-want-your-flight-from-austin-to-come-from-in-regards-to-average-delay-time.-so-first-i-created-a-simple-scatterplot-between-airport-of-origin-and-delay-time-to-see-if-any-trends-existed.}}

\includegraphics{Airport_files/figure-latex/unnamed-chunk-4-1.pdf}

\hypertarget{it-appears-that-a-few-airports-have-longer-average-delay-times-than-others.-so-i-calculated-the-average-delay-time-in-minutes-for-every-airport-of-origin-and-i-created-a-bar-graph-to-visualize-average-delay-time-in-austin-from-airport-of-origin-in-descending-order.}{%
\subsubsection{It appears that a few airports have longer average delay
times than others. So, I calculated the average delay time in minutes
for every airport of origin, and I created a bar graph to visualize
average delay time in Austin from airport of origin in descending
order.}\label{it-appears-that-a-few-airports-have-longer-average-delay-times-than-others.-so-i-calculated-the-average-delay-time-in-minutes-for-every-airport-of-origin-and-i-created-a-bar-graph-to-visualize-average-delay-time-in-austin-from-airport-of-origin-in-descending-order.}}

\includegraphics{Airport_files/figure-latex/unnamed-chunk-6-1.pdf}

\hypertarget{it-looks-like-you-dont-want-your-flight-to-originate-from-knoxville-tys-oklahoma-city-okc-or-philidelphia-phl-as-those-have-an-average-delay-time-between-80-and-95-minutes-when-flying-to-the-austin-airport.}{%
\subsubsection{It looks like you don't want your flight to originate
from Knoxville (TYS), Oklahoma City (OKC), or Philidelphia (PHL) as
those have an average delay time between 80 and 95 minutes when flying
to the Austin
airport.}\label{it-looks-like-you-dont-want-your-flight-to-originate-from-knoxville-tys-oklahoma-city-okc-or-philidelphia-phl-as-those-have-an-average-delay-time-between-80-and-95-minutes-when-flying-to-the-austin-airport.}}

\hypertarget{question-2}{%
\section{Question 2}\label{question-2}}

\hypertarget{for-subclass-350}{%
\subsection{For Subclass 350}\label{for-subclass-350}}

\hypertarget{for-starters-lets-take-a-look-at-what-the-data-looks-like-for-the-350-subclass.}{%
\subsubsection{For starters let's take a look at what the data looks
like for the 350
subclass.}\label{for-starters-lets-take-a-look-at-what-the-data-looks-like-for-the-350-subclass.}}

\includegraphics{Airport_files/figure-latex/unnamed-chunk-8-1.pdf}

\includegraphics{Airport_files/figure-latex/unnamed-chunk-10-1.pdf}

\hypertarget{what-we-see-in-this-graph-shows-that-as-k-increases-rmse-of-the-out-of-sample-prediction-decreases.-what-i-suspect-is-going-on-is-that-most-of-the-data-in-this-set-for-the-550-subclass-is-for-cars-with-less-than-50000-miles.-also-the-data-in-the-range-of-0-to-50000-miles-doesnt-have-an-obvious-trend-in-the-price-range-excluding-the-new-cars-have-a-higher-price.-so-the-best-prediction-in-kmeans-for-the-price-of-the-car-given-how-many-miles-driven-it-has-in-the-550-subclass-might-end-up-being-the-sample-average.}{%
\subsubsection{What we see in this graph shows that as k increases, rmse
of the out of sample prediction decreases. What I suspect is going on is
that most of the data in this set for the 550 subclass is for cars with
less than 50,000 miles. Also, the data in the range of 0 to 50,000 miles
doesn't have an obvious trend in the price range (excluding the new cars
have a higher price). So, the best prediction in kmeans for the price of
the car given how many miles driven it has in the 550 subclass might end
up being the sample
average.}\label{what-we-see-in-this-graph-shows-that-as-k-increases-rmse-of-the-out-of-sample-prediction-decreases.-what-i-suspect-is-going-on-is-that-most-of-the-data-in-this-set-for-the-550-subclass-is-for-cars-with-less-than-50000-miles.-also-the-data-in-the-range-of-0-to-50000-miles-doesnt-have-an-obvious-trend-in-the-price-range-excluding-the-new-cars-have-a-higher-price.-so-the-best-prediction-in-kmeans-for-the-price-of-the-car-given-how-many-miles-driven-it-has-in-the-550-subclass-might-end-up-being-the-sample-average.}}

\hypertarget{now-for-subclass-65amg}{%
\section{Now for Subclass 65AMG}\label{now-for-subclass-65amg}}

\hypertarget{starting-with-the-plot-of-the-data}{%
\subsubsection{Starting with the plot of the
data}\label{starting-with-the-plot-of-the-data}}

\includegraphics{Airport_files/figure-latex/unnamed-chunk-12-1.pdf}

\hypertarget{contrast-this-scatterplot-of-mileage-against-price-for-65amg-with-the-550-with-the-corresponding-scatterplot-for-the-550-subclass.-there-is-a-trend-that-we-can-see.-the-higher-the-mileage-the-loweer-the-the-price-tends-to-be.-there-isnt-a-clump-of-data-in-this-data.}{%
\subsubsection{Contrast this scatterplot of mileage against price for
65AMG with the 550 with the corresponding scatterplot for the 550
subclass. There is a trend that we can see. The higher the mileage, the
loweer the the price tends to be. There isn't a ``clump'' of data in
this
data.}\label{contrast-this-scatterplot-of-mileage-against-price-for-65amg-with-the-550-with-the-corresponding-scatterplot-for-the-550-subclass.-there-is-a-trend-that-we-can-see.-the-higher-the-mileage-the-loweer-the-the-price-tends-to-be.-there-isnt-a-clump-of-data-in-this-data.}}

\includegraphics{Airport_files/figure-latex/unnamed-chunk-14-1.pdf}

\hypertarget{just-as-expected-there-does-seem-to-be-a-cutoff-for-optimal-value-of-k-for-the-out-of-sample-rmse-prediction-for-the-65amg-model.-lets-graph-this-again-but-this-time-zoomed-in-on-values-of-k-between-150-220}{%
\subsubsection{Just as expected, there does seem to be a cutoff for
optimal value of k for the out of sample RMSE prediction for the 65AMG
model. Let's graph this again, but this time ``zoomed in'' on values of
k between
150-220}\label{just-as-expected-there-does-seem-to-be-a-cutoff-for-optimal-value-of-k-for-the-out-of-sample-rmse-prediction-for-the-65amg-model.-lets-graph-this-again-but-this-time-zoomed-in-on-values-of-k-between-150-220}}

\includegraphics{Airport_files/figure-latex/unnamed-chunk-16-1.pdf}

\hypertarget{the-optimal-k-tends-to-be-somewhere-between-200-220-for-the-65amg-subclass.it-just-depends-on-what-data-ends-up-being-in-the-test-set.}{%
\subsubsection{The optimal K tends to be somewhere between 200-220 for
the 65AMG Subclass.It just depends on what data ends up being in the
test
set.}\label{the-optimal-k-tends-to-be-somewhere-between-200-220-for-the-65amg-subclass.it-just-depends-on-what-data-ends-up-being-in-the-test-set.}}


\end{document}
